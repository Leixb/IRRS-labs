\section{Computing tf-idf ’s and cosine similarity}

We implemented the code using \texttt{numpy} and tested it against the \texttt{docs} examples, obtaining
the same results we computed in class.

\subsection{Finding related documents}

Once we had the TF-IDF computation properly implemented, we decided to implement a program that could
provide us with related documents in the same index. Our goal was that, given a document in the index,
suggest the top $N$ documents with the highest similarity score.

The implementation is quite simple, but since we wanted it to be fast, we decided to use sliced scroll
queries from ElasticSearch in order to process the computations in parallel \cite{noauthor_paginate_nodate}.
Since we are only interested on the top $N$ most common documents, we can use a heap queue
with limited size along with python generators to only save in memory $N$ results. With this parallel design,
we can fetch the most similar documents in less than 15 seconds on a 12 thread machine. By comparison,
the same program on a single thread takes 90 seconds.\footnote{Note: this same parallel optimizations were also
applied on the \texttt{CountWords} and \texttt{IndexFilesPreprocess} for the previous sections with similar speedups.
This allowed us to test all the combinations we discussed previously in reasonable times.
The parallel implementation of these files are in the deliverable with the suffix \texttt{\_par}}

Below we show an example query for a document in the cryptology newsgroups. It outputs the scores for each document
along with the top tf-idf terms of the original document:
\begin{minted}{shell-session}
$./TFIDF_experiment.py --index "20_newsgroups-lowercase-asciifolding-stop-snowball" $DATA/20_newsgroups/sci.crypt/0011681
Original file:	data/20_newsgroups/sci.crypt/0011681
Keywords: des, send, recipi, softwar, encrypt, legal, 1993apr22.125402.27561, fripp.ri.cadre.com, ri.cadre.com, licens
1.0000 data/20_newsgroups/sci.crypt/0011681
0.5683 data/20_newsgroups/sci.crypt/0011625
0.3514 data/20_newsgroups/sci.crypt/0011676
0.3494 data/20_newsgroups/sci.crypt/0011109
0.3455 data/20_newsgroups/sci.crypt/0011599
0.3285 data/20_newsgroups/sci.crypt/0011708
0.3184 data/20_newsgroups/sci.crypt/0011620
0.2896 data/20_newsgroups/talk.religion.misc/0019137
0.2875 data/20_newsgroups/sci.crypt/0011044
0.2864 data/20_newsgroups/sci.crypt/0011639
\end{minted}

If we inspect these documents, all of them talk about DES \cite{noauthor_data_2022} a Data encryption
standard which was cracked in 1997. Even the one from talk.religion.misc document is a discussion about it.
If we instruct our program to ignore paths from \texttt{sci}, we get some more talks in religious forums
and \texttt{misc.forsale/0006636}, which is a posting of someone who wants an airplane ticket
from \emph{\textbf{Des} Moines}, which explains the high score of the document. One thing to consider is that
many of the newsgroups documents are email threads, where the replies copy the original message on top. In most
cases, the related documents are part of the same mail chain.


% When applied to a single document, this related documents is usefull for a recommender, but if we apply it
% for all documents in the collection and restrict the similarity, we can obtain a network of realted documents
% and communities


\section{Conclusion}

With this laboratory session, we have learned the basics of performing queries in Elasticsearch
to compute our own metrics. We have been able to compute TF-IDF from basic text statistics in an efficient
manner and analyzed the effects that some tokenizers and filters that ElasticSearch provides
have when indexing a corpus. Additionally, we explored how ElasticSearch queries can be parallelized to
reduce the time it takes to compute our metrics.

The \emph{standard} and \emph{classic} tokenizers behave similarly across our 3 different corpuses,
but the \emph{letter} and \emph{whitespace} behave differently depending on the corpus.

For the filters found that \emph{stop} has a noticeable effect on the most common words as well as the number of total words
of a document, \emph{lowercase} reduces the number of unique words in a document significantly and 
\emph{snowball} is the most aggressive stemmer.

%For this laboratory we had been working directly with Elastic Search through programming it and their characteristic scripts