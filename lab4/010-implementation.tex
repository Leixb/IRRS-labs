%! TEX root = **/report.tex


% DATA: https://github.com/Leixb/IRRS-labs/releases/download/v4.0.0rc/data.tar.gz

% To deliver:
% You will have to deliver a report with the results that you have obtained with the dataset,
% explaining
% how you have generated the datasets,
% what kind of clusters you have found in the data
% and the effect of the size of the vocabulary (number of words selected for representing the documents)
% and the impact of the number of mappers/reducers in the computational cost.
% You can also include in the document the
% main difficulties/choices you had made while implementing this lab session’s

\section*{Introduction}
In this deliverable we will explore the implementation of the \emph{K-Means} algorithm
for clustering similar words through the \emph{MapReduce} computational paradigm.

\section{Implementation}

\subsection{MapReduce KMeansStep.py}

We followed the implementation as outlined in the assignment:

\begin{tikzpicture}[every node/.style={text centered, minimum height=1cm, rectangle, draw}, node distance=0.5cm and 2cm]
    \node (load)  {load\_data};
    \node[right=of load] (assign0)  {assign\_prototype};
    \node[above=of assign0,minimum width=3.7cm] (assign1)  {assign\_prototype};
    \node[draw=none, below=of assign0, node distance=0.1cm and 2cm] (dots)  {$\vdots$};
    \node[below=of dots,minimum width=3.7cm, node distance=0.1cm and 2cm] (assign2)  {assign\_prototype};
    \node[right=of assign0,minimum width=3.7cm] (aggregate)  {\texttt{aggregate\_prototype}};

    \draw[->] (load) -- (assign0);
    \draw[->] (load.10) -- (assign1.west);
    \draw[->] (load.350) -- (assign2.west);
    \draw[->] (assign0) -- (aggregate);
    \draw[->] (assign1.east) -- (aggregate.170);
    \draw[->] (assign2.east) -- (aggregate.190);
\end{tikzpicture}

Where each mapper loads the prototype of each cluster from the provided prototype file and then
computed for each document the Jaccard distance to each prototype. Then it assigns the
prototype with smallest distance to it and emits it to the reducer.

The reducer takes all the documents assigned for each prototype (cluster) and computes the
new prototype from the frequencies of the words in the documents assigned to it.

In our implementation, we used the python data structure \emph{Counter} from the
collections built-in library to efficiently compute the frequencies of the words. This
structure uses a dictionary to store the occurrences of each word.

We considered using two heaps to keep a list of the words and documents sorted as they
are obtained from the mappers. This could theoretically improve the performance
in some cases since we would not have to sort the data at the end of the reducer and
instead the sort is done as the data is obtained. However, we could not justify
this change since the data is not that big and the sorting is not that expensive in our case.

\subsection{MapReduce KMeans.py}

In this script, we performed the background work to run the \emph{KMeans-Step} multiple
times until the prototypes converge.

The only thing we had to do was to save the prototypes in the correct file and format
so that the mappers could load them. And then process the results to check for convergence.

\subsubsection{Convergence}

To check for convergence, our initial approach was to check if the prototypes were the same,
but we quickly realized that floating point precision issues made this approach unreliable.
One workaround we tried was using \emph{numpy.isclose} or \emph{numpy.allclose} to do the
comparison with a small tolerance for numerical errors. This worked but we found that a better
approach was to check if the assigned documents to each consecutive prototype were the same.

To do so, we modified the \emph{KMeans-Step} to emit the list of assigned documents
as well as the prototype for each cluster, then we can check if the set of
assigned documents was the same that the previous prototype. This indicates that the
algorithm has converged.

Another issue was that for some reason, the assigned prototypes swapped names, meaning
that sometimes the cluster 0 would be named 1 and the cluster 1 named 0 in the next iteration.
To solve this, we ignored the cluster names and sorted the list of assigned documents for each
cluster before comparing them.

Given that we had the information of the assigned documents, we decided to saved it into a file
in order to be able to analyze the clusters later.
