%! TEX root = **/report.tex
\section{Results}

Finally, we took a look at the clusters we obtained with our experimentation. As we have seen,
there are 8 different categories in the dataset but there are some categories that are very broad
and probably can be split into more specific categories. Therefore, we started with 20 clusters.

We run a similar script to the one used on the experiments from the previous sections but using
a more limited search space, a higher number of iterations and 20 clusters as the initial
number of clusters.

The initial expectation was to find clusters that encompassed the subcategories in some
of the bigger and broader categories (mainly computer science and astrophysics). However,
we found that in all the cases where the algorithm converged, there were never
more than 3 clusters.

In the cases where the algorithm did not converge, we found that most of the time
there was one or two clusters that were very large and encompassed most of the
documents, this is a weakness of \emph{K-means}. Even then, they did not have more
than 7 clusters (which is less than the number of categories).

The most interesting results were with the frequency range $[0.08,\, 0.1]$ shown
in \ref{tab:results}. As we can see, the first cluster is much larger and takes
most of the documents from all categories. The other smaller clusters have some dominant categories
such as math, quantum physics and computer science for cluster 2 and astrophysics for cluster 3.

Cluster 3 is the most interesting since it takes a lot of the documents from the astrophysics category
and the only other two categories that are significantly represented are the two from HEP (High energy
physics) which are very related to astrophysics.

\begin{table}[H]
	\caption{Distribution of documents of each category through the cluster}%
	\label{tab:results}%
	\begin{tabular}{lS[table-format=2.2]S[table-format=2.2]S[table-format=2.2]S[table-format=2.2]}
		\toprule
		\multirow{2}{*}{\textbf{Category}} & \multicolumn{4}{c}{\textbf{Cluster}}                        \\
		                                   & {1}                                    & {2}   & {3}   & {4}  \\
		\midrule
		% astro-ph & 7414   & 311    & 4395   & 939    \\
		% cond-mat & 4451   & 265    & 15     & 300    \\
		% cs       & 15909  & 1979   & 14     & 963    \\
		% hep-ph   & 1308   & 82     & 111    & 108    \\
		% hep-th   & 1282   & 66     & 40     & 71     \\
		% math     & 5525   & 1056   & 12     & 370    \\
		% physics  & 5709   & 565    & 50     & 518    \\
		% quant-ph & 1562   & 223    & 2      & 86     \\
		astro-ph                           & 56.77                                  & 2.38  & 33.65 & 7.19 \\
		cond-mat                           & 88.47                                  & 5.27  & 0.30  & 5.96 \\
		cs                                 & 84.33                                  & 10.49 & 0.07  & 5.10 \\
		hep-ph                             & 81.29                                  & 5.10  & 6.90  & 6.71 \\
		hep-th                             & 87.87                                  & 4.52  & 2.74  & 4.87 \\
		math                               & 79.35                                  & 15.17 & 0.17  & 5.31 \\
		physics                            & 83.44                                  & 8.26  & 0.73  & 7.57 \\
		quant-ph                           & 83.40                                  & 11.91 & 0.11  & 4.59 \\
		\bottomrule
	\end{tabular}
\end{table}

In \cref{tab:vocabulary} we show some of the words that are used in this execution (out of 100). We can
see that there are some field specific words such as ``galaxy'' and ``neural'' but also other
more generic words such as ``best'', or ``common''.

\begin{table}[H]
	\caption{Vocabulary for frequency range $[0.08,\, 0.1]$ (top 20)}%
	\label{tab:vocabulary}%
	\begin{tabular}{rl}
		\toprule
		Occurrences & Word     \\
		\midrule
		4946        & select   \\
		4946        & constant \\
		4920        & neural   \\
		4912        & shown    \\
		4879        & address  \\
		4857        & key      \\
		4824        & spatial  \\
		4769        & art      \\
		4745        & reveal   \\
		4723        & us       \\
		\bottomrule
	\end{tabular}
	\hspace{2em}
	\begin{tabular}{rl}
		\toprule
		Occurrences & Word    \\
		\midrule
		4722        & open    \\
		4710        & galaxi  \\
		4704        & complet \\
		4675        & part    \\
		4665        & finit   \\
		4642        & best    \\
		4639        & contain \\
		4632        & probabl \\
		4628        & common  \\
		4604        & signal  \\
		\bottomrule
	\end{tabular}
\end{table}

% data/arxiv_abs_30/100_0.08_0.1/vocabulary.txt

% data/arxiv_abs_30/100_0.04_0.07/vocabulary.txt
